\documentclass[a4paper, 12pt]{article}
\usepackage{graphicx}
\usepackage{amsmath}
\usepackage{hyperref}
\usepackage{color}
\usepackage{gensymb}
% Header and Footer
\usepackage{fancyhdr}
\pagestyle{fancy}
\fancyhf{}
\fancyhead[L]{\textbf{Memo}}
\fancyhead[R]{\textbf{\thepage}}

% Title Section
\title{Transient Receptor Potential V1 and Mathematical model}
\author{TaeYoon Kim}
\date{\today}

\begin{document}

\maketitle

% Summary Section
\section*{Summary}
\noindent
This research is about mathematical modeling of TRPV1(Transient Receptor Potential V1).\\
By modeling TRPV1, we can understand ion dynamics of TRPV1.
\vspace{1cm}

% Sections for Organizing Thoughts
\section*{Transient Receptor Potential V1}
The transient receptor potential cation channel subfamily V member 1(TRPV1),
also known as the {\color{red}capsaicin receptor} and the vanilloid receptor 1. 
TRPV1 is an element of or mechanism used by the mammailan somatosensory system.
It is a nonselective cation channel that may be activated by
a wide variety of exogenous and endogenous physical and chemical stimuli.
The best-known activators of TRPV1 are:
temperature greater than 43{\degree}C,
capsaicin, and allylisothiocyanate.
The activation of TRPV1 leads to a painful, burning sensation.
{\color{blue}TPRV1 receptors are found mainly in the nociceptive neurons of the 
peripheral nervous system.}
TRPV1 is involved in the transmission and modulation of pain,
as well as the integration of diverse painful stimuli.

\subsection*{Section 1: Idea 1}
\noindent
\textbf{Description:} \\
Provide a detailed description of the first idea.

\vspace{0.5cm}

\noindent
\textbf{Key Points:}
\begin{itemize}
    \item Key point 1
    \item Key point 2
    \item Key point 3
\end{itemize}

\vspace{0.5cm}

\noindent
\textbf{Notes:} \\
Additional notes and reflections on the first idea.

\vspace{1cm}

\subsection*{Section 2: Idea 2}
\noindent
\textbf{Description:} \\
Provide a detailed description of the second idea.

\vspace{0.5cm}

\noindent
\textbf{Key Points:}
\begin{itemize}
    \item Key point 1
    \item Key point 2
    \item Key point 3
\end{itemize}

\vspace{0.5cm}

\noindent
\textbf{Notes:} \\
Additional notes and reflections on the second idea.

\vspace{1cm}

\subsection*{Section 3: Idea 3}
\noindent
\textbf{Description:} \\
Provide a detailed description of the third idea.

\vspace{0.5cm}

\noindent
\textbf{Key Points:}
\begin{itemize}
    \item Key point 1
    \item Key point 2
    \item Key point 3
\end{itemize}

\vspace{0.5cm}

\noindent
\textbf{Notes:} \\
Additional notes and reflections on the third idea.

\vspace{1cm}

% Action Plan Section
\section*{Action Plan}
\noindent
Based on the organized thoughts, outline a clear action plan.

\vspace{0.5cm}

\noindent
\textbf{Immediate Actions:}
\begin{enumerate}
    \item Immediate action 1
    \item Immediate action 2
    \item Immediate action 3
\end{enumerate}

\vspace{0.5cm}

\noindent
\textbf{Long-term Actions:}
\begin{enumerate}
    \item Long-term action 1
    \item Long-term action 2
    \item Long-term action 3
\end{enumerate}

\vspace{0.5cm}

\noindent
\textbf{Follow-up:} \\
Outline steps for follow-up and ensuring the action plan is executed.

\end{document}